\subsection{Oppgave 1}

Den logiske kretsen er en latch. Denne kan lagre en bit med informasjon. Lagringen skjer ved at input-signalet A sendes ut til output. B er en lås. Det vil si at når B skrus av beholder outputen den verdien den tidligere hadde inntil B skrus på igjen. På denne måten kan vi aktivere B, sende et signal på A, deaktivere B, og senere lese av hva A var da vi lagret den.

\begin{figure}[H]
    \includegraphics[width=0.8\textwidth]{../imgs/week5-task1.png}
    \caption{B er skrudd av, og output er låst. Derfor er det mulig for output å være motsatt av A.}
\end{figure}